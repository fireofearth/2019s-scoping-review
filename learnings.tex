\documentclass[12pt]{extarticle}
\usepackage[english]{babel}
\usepackage{graphicx}
\usepackage{framed}
\usepackage[normalem]{ulem}
\usepackage{amsmath}
\usepackage{amsthm}
\usepackage{amssymb}
\usepackage{amsfonts}
\usepackage{enumerate}
\usepackage{bm}
\usepackage[utf8]{inputenc}
\usepackage[top=1 in,bottom=1in, left=1 in, right=1 in]{geometry}

\usepackage{hyperref}
\hypersetup{
    colorlinks=true,
    linkcolor=blue,
    filecolor=magenta,      
    urlcolor=cyan,
}
\urlstyle{same}

\newcommand{\cvec}[1]{{\mathbf #1}}
\newcommand{\rvec}[1]{\vec{\mathbf #1}}
\newcommand{\ihat}{\hat{\textbf{\i}}}
\newcommand{\jhat}{\hat{\textbf{\j}}}
\newcommand{\khat}{\hat{\textbf{k}}}
\newcommand{\minor}{{\rm minor}}
\newcommand{\trace}{{\rm trace}}
\newcommand{\spn}{{\rm Span}}
\newcommand{\rem}{{\rm rem}}
\newcommand{\ran}{{\rm range}}
\newcommand{\range}{{\rm range}}
\newcommand{\mdiv}{{\rm div}}
\newcommand{\proj}{{\rm proj}}
\newcommand{\<}{\langle}
\renewcommand{\>}{\rangle}
\renewcommand{\emptyset}{\varnothing}
\newcommand{\attn}[1]{\textbf{#1}}
\theoremstyle{definition}
\newtheorem{theorem}{Theorem}
\newtheorem{corollary}{Corollary}
\newtheorem*{definition}{Definition}
\newtheorem*{example}{Example}
\newtheorem*{note}{Note}
\newtheorem{exercise}{Exercise}
\newcommand{\bproof}{\bigskip {\bf Proof. }}
\newcommand{\eproof}{\hfill\qedsymbol}
\newcommand{\Disp}{\displaystyle}
\newcommand{\qe}{\hfill\(\bigtriangledown\)}
\setlength{\columnseprule}{1 pt}

\include{./notation}

\title{Notes for 2019S}
\author{Colin}
\date{May 2019}

\begin{document}

\maketitle

\section{Numerical Analysis}
\subsection{Poisson solver}
\href{http://users.cs.northwestern.edu/~jet/Teach/2004_3spr_IBMR/poisson.pdf}{Poisson Solvers}.
We are trying to solve problems of the form $-a\nabla^2 +\bm{b}\cdot \nabla u + cu = f$. This can be solved using \href{https://github.com/ginkgo-project/ginkgo/wiki/Tutorial:-Building-a-2D-Poisson-Solver}{Gingko Solver}

\subsection{Bulirsch Stoer}
Obtains numerical solutions to ODEs.
\href{https://people.sc.fsu.edu/~sshanbhag/BulirschStoer.pdf}{https://people.sc.fsu.edu/~sshanbhag/BulirschStoer.pdf}
\href{http://web.mit.edu/ehliu/Public/Spring2006/18.304/implementation_bulirsch_stoer.pdf}{Article}
\href{https://en.wikipedia.org/wiki/Bulirsch%E2%80%93Stoer_algorithm}{wikipedia article}

\subsection{Adam Bashforth}
Linear multistep method for ODEs
\href{https://en.wikipedia.org/wiki/Linear_multistep_method#Adams%E2%80%93Bashforth_methods}{wikipedia article}

\end{document}
